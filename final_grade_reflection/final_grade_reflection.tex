% Options for packages loaded elsewhere
\PassOptionsToPackage{unicode}{hyperref}
\PassOptionsToPackage{hyphens}{url}
\PassOptionsToPackage{dvipsnames,svgnames,x11names}{xcolor}
%
\documentclass[
  letterpaper,
  DIV=11,
  numbers=noendperiod]{scrartcl}

\usepackage{amsmath,amssymb}
\usepackage{lmodern}
\usepackage{iftex}
\ifPDFTeX
  \usepackage[T1]{fontenc}
  \usepackage[utf8]{inputenc}
  \usepackage{textcomp} % provide euro and other symbols
\else % if luatex or xetex
  \usepackage{unicode-math}
  \defaultfontfeatures{Scale=MatchLowercase}
  \defaultfontfeatures[\rmfamily]{Ligatures=TeX,Scale=1}
\fi
% Use upquote if available, for straight quotes in verbatim environments
\IfFileExists{upquote.sty}{\usepackage{upquote}}{}
\IfFileExists{microtype.sty}{% use microtype if available
  \usepackage[]{microtype}
  \UseMicrotypeSet[protrusion]{basicmath} % disable protrusion for tt fonts
}{}
\makeatletter
\@ifundefined{KOMAClassName}{% if non-KOMA class
  \IfFileExists{parskip.sty}{%
    \usepackage{parskip}
  }{% else
    \setlength{\parindent}{0pt}
    \setlength{\parskip}{6pt plus 2pt minus 1pt}}
}{% if KOMA class
  \KOMAoptions{parskip=half}}
\makeatother
\usepackage{xcolor}
\setlength{\emergencystretch}{3em} % prevent overfull lines
\setcounter{secnumdepth}{-\maxdimen} % remove section numbering
% Make \paragraph and \subparagraph free-standing
\ifx\paragraph\undefined\else
  \let\oldparagraph\paragraph
  \renewcommand{\paragraph}[1]{\oldparagraph{#1}\mbox{}}
\fi
\ifx\subparagraph\undefined\else
  \let\oldsubparagraph\subparagraph
  \renewcommand{\subparagraph}[1]{\oldsubparagraph{#1}\mbox{}}
\fi


\providecommand{\tightlist}{%
  \setlength{\itemsep}{0pt}\setlength{\parskip}{0pt}}\usepackage{longtable,booktabs,array}
\usepackage{calc} % for calculating minipage widths
% Correct order of tables after \paragraph or \subparagraph
\usepackage{etoolbox}
\makeatletter
\patchcmd\longtable{\par}{\if@noskipsec\mbox{}\fi\par}{}{}
\makeatother
% Allow footnotes in longtable head/foot
\IfFileExists{footnotehyper.sty}{\usepackage{footnotehyper}}{\usepackage{footnote}}
\makesavenoteenv{longtable}
\usepackage{graphicx}
\makeatletter
\def\maxwidth{\ifdim\Gin@nat@width>\linewidth\linewidth\else\Gin@nat@width\fi}
\def\maxheight{\ifdim\Gin@nat@height>\textheight\textheight\else\Gin@nat@height\fi}
\makeatother
% Scale images if necessary, so that they will not overflow the page
% margins by default, and it is still possible to overwrite the defaults
% using explicit options in \includegraphics[width, height, ...]{}
\setkeys{Gin}{width=\maxwidth,height=\maxheight,keepaspectratio}
% Set default figure placement to htbp
\makeatletter
\def\fps@figure{htbp}
\makeatother

\KOMAoption{captions}{tableheading}
\makeatletter
\makeatother
\makeatletter
\makeatother
\makeatletter
\@ifpackageloaded{caption}{}{\usepackage{caption}}
\AtBeginDocument{%
\ifdefined\contentsname
  \renewcommand*\contentsname{Table of contents}
\else
  \newcommand\contentsname{Table of contents}
\fi
\ifdefined\listfigurename
  \renewcommand*\listfigurename{List of Figures}
\else
  \newcommand\listfigurename{List of Figures}
\fi
\ifdefined\listtablename
  \renewcommand*\listtablename{List of Tables}
\else
  \newcommand\listtablename{List of Tables}
\fi
\ifdefined\figurename
  \renewcommand*\figurename{Figure}
\else
  \newcommand\figurename{Figure}
\fi
\ifdefined\tablename
  \renewcommand*\tablename{Table}
\else
  \newcommand\tablename{Table}
\fi
}
\@ifpackageloaded{float}{}{\usepackage{float}}
\floatstyle{ruled}
\@ifundefined{c@chapter}{\newfloat{codelisting}{h}{lop}}{\newfloat{codelisting}{h}{lop}[chapter]}
\floatname{codelisting}{Listing}
\newcommand*\listoflistings{\listof{codelisting}{List of Listings}}
\makeatother
\makeatletter
\@ifpackageloaded{caption}{}{\usepackage{caption}}
\@ifpackageloaded{subcaption}{}{\usepackage{subcaption}}
\makeatother
\makeatletter
\@ifpackageloaded{tcolorbox}{}{\usepackage[many]{tcolorbox}}
\makeatother
\makeatletter
\@ifundefined{shadecolor}{\definecolor{shadecolor}{rgb}{.97, .97, .97}}
\makeatother
\makeatletter
\makeatother
\ifLuaTeX
  \usepackage{selnolig}  % disable illegal ligatures
\fi
\IfFileExists{bookmark.sty}{\usepackage{bookmark}}{\usepackage{hyperref}}
\IfFileExists{xurl.sty}{\usepackage{xurl}}{} % add URL line breaks if available
\urlstyle{same} % disable monospaced font for URLs
\hypersetup{
  pdftitle={Final Grade Reflection},
  pdfauthor={Sebastian Boksem},
  colorlinks=true,
  linkcolor={blue},
  filecolor={Maroon},
  citecolor={Blue},
  urlcolor={Blue},
  pdfcreator={LaTeX via pandoc}}

\title{Final Grade Reflection}
\author{Sebastian Boksem}
\date{}

\begin{document}
\maketitle
\ifdefined\Shaded\renewenvironment{Shaded}{\begin{tcolorbox}[frame hidden, sharp corners, interior hidden, boxrule=0pt, enhanced, borderline west={3pt}{0pt}{shadecolor}, breakable]}{\end{tcolorbox}}\fi

(I am having issues rendering the document with TeX. I have tried
installing it but it told me TeX Live is not on path. I've used a
separate .tex to PDF converter and uploaded it)

After taking into account the first six weeks of my STATS 331, I would
consider giving myself a grade of an A-. I think this decision stems
from my ability to try and fully learn the targets attached,. As a
result, I think my learning has followed in other aspects such as
revising my thinking and being collaborative in my group, peer reviews,
and Discord. Over these last 6 weeks, I feel like I have ended up being
able to successfully checkoff most learning targets. For example, with
working with data, I have successfully been able to filter rows from a
data base for a variety of data types. I think a clear example of this
lies in my Lab 3 submission, Hip-Hop. Under the Finding Bieber question,
I was able to find a single participant out of 10752 rows by different
uses of grouping and filtering both factor and numbers. Not only this,
in the process of filtering these factor variables, I needed to turn
them into factors originally, which was accomplished and done in a
program efficiency way. Instead of creating a mutate with changing the
variable to a factor for all the variables I needed to do, I instead
followed to create conciser code by including an across function (and
did that in the labs that followed). I feel like I also accomplished
parts of ``Working with Data'' by the use of different joins, as well as
filtering joins inside more specifically my Lab 4: Avocado Prices. In
this lab, A key factor was the use of filtering joins to remove certain
aspects of regions that were not needed. My ability to filter out
certain regions has clearly been shown in the clean\_avocado data set,
which I was able to remove rows that were not connected to any metro
area (and following create new data sets surrounding US total in
addition to California regions) As a result of this, I was further able
to analyze and make visualizations about their respective Avocados
sales. I believe this also demonstrates my ability to be able to cycle
between the use of filtering joins and a filter() statement. These use
of working with data through the use of joins also connects to the
learning targets associated with ``Data Visualization and
Summarization''. In Lab4 surrounding the mean average price for
different regions within California, I calculated numerical summaries of
variables, then following that, I further used a left join to connect
back to the initial data after grouping to create different
visualizations such as a box plot. I feel like this highlights my
abilities to creates these summaries, while also highlighting my ability
to differentiate between what kind of joins exists and use each one in
their own unique way.

I have completed all the challenge assignments to date in order to
highlight my creative ability in creating these data visualizations. I
feel like a good example in which outside learning was used could stem
from the Lab 4 challenge problem in which we were attempting to connect
each California Region's housing prices to the average price of an
avocado. I ended up creating a scatter plot showcasing the differences
and found a slight association that as housing prices increases so did
the price of avocados. This helped me overall extend my learning outside
this class by being able to create a data set of my own from outside
resources which previously I did not really know how to do.

Historically, I have completed every reflection attempt given the
opportunity and have received check marks on all labs and challenge
assignments, just waiting on a formatting reflection on Lab4. This class
has overall been great in terms of teamwork for me. My team have done a
really influential job attempting to try and complete the practice
assignments and if one of us ever has any questions about labs we tend
to answer and try to help one another after class hours. For peer
reviews, I have reviewed and given helpful syntax suggestions to every
person I have been assigned to, although I will say at certain times it
can be hard due to us having the same coding knowledge as one another. I
have started to be more active in Discord to answer questions. One
example of this was during Lab4: Avocado Prices in which a person had a
question concerning how to reference a non-syntactically correct column.
I previously spent the last 30 minutes looking into how to solve issues
like that, and seeing there question arose I told them that back ticks
were needed for certain columns like that. In addition, I also just used
the code that they had (even though my code for the problem was
different) and adjusting it with the back ticks to actually see if it
contained the correct answer for them. I think this was important as
rather then just copying and pasting my own code, it allowed them to use
that small knowledge gained from me to get closer to the answer to the
problem on there own.

One thing I believe that would negatively affect my grade and would note
for improvement over these next three weeks is my ability to answer the
practice assignments on time. To my knowledge, I believe I have turned
in two practice assignments later than when they were suppose to be due.
To be honest, I think a small part of it has to be wanting to give
myself a small break throughout the week whereas the the weekend and
start of the week I would be working on the labs, challenge assignments,
and the preview activities immensly. As a result, I found my self a
little burnt out and waited until the weekend to complete these
activities before the lab work. This is something I hope to improve on
in the last part of the quarter and have started to designate times on
Wednesday to complete the assignments. This class has been one of the
most challenging, but at the same time rewarding classes I have taken. A
goal I had from the start of the quarter was to understand R in a
greater ability than I did in previous classes. I think as of now I have
accomplished that immensely and hope to continue to do so until the end
of the quarter!



\end{document}
